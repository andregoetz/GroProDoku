\chapter{Abweichung und Ergänzung}\label{ch:abweichung-und-ergaenzung}

Im Zuge der Konzeption wurden Entscheidungen getroffen, von welchen die tatsächliche Umsetzung abweicht, um eine bessere Laufzeit oder auch Funktion zu gewährleisten.
Dazu zählt vor allem die Gesamtzahl der Threads der einzelnen EVA-Komponenten.
Die Überlegung, dass es mehrere Einlese-Threads und Verarbeitungs-Threads gibt, basierte auf der Annahme, dass das Einlesen der Dateien deutlich länger brauchen würde, sodass eine Bereitstellung von 20Hz nicht erfolgen könnte.
Dazu sei allerdings gesagt, dass es für spätere Zwecke im~\nameref{sec:ausblick} geplant ist, dies zu implementieren, wie bereits für die Ausgabe geschehen.\\
Weitere Änderungen befinden sich im Klassendiagramm.
So wurde beispielsweise die Klasse \enquote{Pulsbreite} zu einem Record verändert, die Klasse \enquote{Datenpunkt} hinzugefügt und Felder des Models \enquote{Messdaten} von Maps zu Listen an Datenpunkten verändert.
Es wurden außerdem einige Methoden zu den Klassen hinzugefügt, um eine bessere Übersicht des Codes zu gewährleisten.

