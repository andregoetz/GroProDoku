\begin{titlepage}
	% TODO: wird das geometry Paket genutzt statt der Mechanismen aus KOMA-Skript, kann die folgende Zeile z.B. durch "\newgeometry{...}" ersetzt werden
	\typearea{100} % DVI auf 100 setzen für Titel (kleine Margins)
	\setlength{\parindent}{0pt} % keine Einrückung bei neuen Absätzen auf dieser Seite

	\begin{flushright}
		\includegraphics[height=5cm]{images/FH-Aachen-r_svg-raw}
	\end{flushright}
	
	\vspace*{-2.5cm}

	\begin{center}
		\textbf{\Huge Fachhochschule~Aachen}

		\vspace*{0.5cm}
		
		\textbf{\Huge Studienort Köln}

		\vspace*{0.75cm}

		{\normalsize\doublespacing Fachbereich~9:~Medizintechnik~und~Technomathematik\\	Studiengang:~Angewandte~Mathematik~und~Informatik}

		\vspace*{2.5cm} % TODO: bei mehr verwendeten Zeilen für den Titel verringern
		
		\begin{minipage}[t]{14cm} % TODO: abhängig von dem konkreten Titel muss diese Breite eventuell angepasst werden für einen passenden Zeilenumbruch
			\begin{center}
				\textbf{\Huge \titleDocument}
			\end{center}
		\end{minipage}
	
		\vspace*{2.5cm} % TODO: bei mehr verwendeten Zeilen für den Titel verringern (symmetrisch zu oben)
		
		\textbf{\Large \subjectDocument}

		\vspace*{0.5cm}
		
		{\normalsize von}
		
		\vspace*{0.5cm}
		
		\textbf{\Large \authorDocument}
	
		\vspace*{3.5cm}
		
		\begin{minipage}[t]{13cm}
			\begin{center}
				\begin{tabular}{llll}
					Matrikelnummern: & 3237534 \\
				\end{tabular}
			\end{center}
		\end{minipage}
		
		\vspace*{2.5cm}
	
		{\large \locationDocument, den \dateDocument}
	\end{center}
\end{titlepage}
