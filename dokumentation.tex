\documentclass[
    12pt, % Schriftgröße
    oneside, % zweiseitiger Modus
    ngerman, % deutsches Dokument
    BCOR=0mm, % Bindungskorrektur
    DIV=10 % Division (Anzahl Spalten/Zeilen pro Seite, bestimmt implizit Margins)
]{scrreprt}

\newcommand{\ArbeitTitelseite}{Dokumentation der Praktischen Arbeit\ zur Prüfung zum\ Mathematisch-technischen Softwareentwickler}
\newcommand{\titleDocument}{\ArbeitTitelseite}

\newcommand{\Autor}{André Götz}
\newcommand{\authorDocument}{\Autor}

\newcommand{\ArbeitKopfzeile}{Dokumentation der Praktischen Arbeit 2023}
\newcommand{\ArbeitThema}{Bestimmung der Minimalanzahl von Antennen}
\newcommand{\Pruefungsnummer}{TODO}
\newcommand{\Programmiersprache}{Java}
\newcommand{\Compiler}{Maven}
\newcommand{\Rechner}{Lenovo Ideapad Gaming 3 15ACH6}
\newcommand{\CPU}{AMD Ryzen 5 5600H @ 3.30GHz}
\newcommand{\Betriebssystem}{EndeavourOS Linux x86\_64}

\newcommand{\subjectDocument}{Großprogrammierung}
\newcommand{\locationDocument}{Köln}
\newcommand{\dateDocument}{\today}

\title{\titleDocument}
\author{\authorDocument}
\date{\dateDocument}

%Style importieren:
\usepackage{dokumentation}


\begin{document}
    % ============ Anfang =============
    % Titelseite
    \include{title}
    \begingroup
        \hypersetup{hidelinks}
        \tableofcontents
    \endgroup

    % =========== Zahlenteil ===========
    \include{chapters/1-Aufgabenanalyse}
    \include{chapters/2-Verfahrensbeschreibung}
    \chapter{Programmbeschreibung}\label{ch:programmbeschreibung}

\section{Struktogramm}\label{sec:pap}
\begin{center}
    \makebox[\textwidth]{\includegraphics[width=\paperwidth]{GroPro-Leonhard}}
\end{center}

\section{Entwicklungsdokumentation}\label{sec:entwicklerdokumentation}
Die Dokumentation des Programms wurde in Javadoc vorgenommen und kann im Ordner javadoc eingesehen werden.
Hierzu kann die index.html aufgerufen werden

%    \addtocontents{toc}{\protect\newpage}
    \include{chapters/4-Testdokumentation}
    \chapter{Zusammenfassung und Ausblick}\label{ch:zusammenfassung-und-ausblick}


\section{Zusammenfassung}\label{sec:zusammenfassung}

\section{Ausblick}\label{sec:ausblick}

    % ============= Buchstabenteil ==============
    \renewcommand{\thechapter}{\Alph{chapter}}%
    \setcounter{chapter}{0}
    \chapter{Abweichung und Ergänzung}\label{ch:abweichung-und-ergaenzung}

    \chapter{Benutzeranleitung}\label{ch:benutzeranleitung}


\section{Vorbereiten des Systems}\label{sec:vorbereiten-des-systems}

\subsection{Systemvoraussetzungen}\label{subsec:systemvoraussetzungen}
Um sicherzustellen, dass das Programm lauffähig ist, sollte~\Betriebssystem~als Betriebssystem genutzt werden.
Es ist außerdem eine JRE oder JDK in der Version 19 oder höher vonnöten, um das Programm auszuführen.\\
Falls eine erneute Kompilierung des Programms gewünscht ist, empfiehlt es sich die JDK anstelle der JRE zu installieren.
Um diese JRE/JDK anschließend zu nutzen, muss das bin-Verzeichnis dieser in die PATH-Umgebungsvariable hinzugefügt werden.
\subsection{Installation}\label{subsec:installation}
Es ist keine Installation nötig, es reicht das~.zip-Verzeichnis zu entpacken.

\section{Programmaufruf}\label{sec:programmaufruf}
Nach dem Entpacken kann das Programm über den Befehl
\begin{center}
    \colorbox{gray!20}{
        \begin{minipage}{0.9\textwidth}
            java -jar groprosim-andre-goetz.jar "beispiele/bsp1.txt"
        \end{minipage}
    }
\end{center}
in beliebiger Shell ausgeführt werden.
\section{Testen der Beispiele}\label{sec:testen-der-beispiele}
Das Ausführen der automatischen Tests erfolgt über die Datei \enquote{run-examples.sh}.
In der Konsole wird nun das Programm für jede .txt-Datei im Ordner \enquote{beispiele} ausgeführt.
Alle Dateiausgaben befinden sich im Anschluss im selben Ordner mit entsprechenden Endungen (.out, .dem, .dat).

\section{Kompilieren}\label{sec:kompilieren}
Zum Erzeugen der~.jar-Datei sollte Maven genutzt werden.
Der Einfachheit halber muss das bin-Verzeichnis der Maveninstallation ebenfalls in der PATH-Umgebungsvariable aufgenommen werden.
Anschließend lässt sich das Programm kompilieren, indem der Befehl
\begin{center}
    \colorbox{gray!20}{
        \begin{minipage}{0.9\textwidth}
            mvn package
        \end{minipage}
    }
\end{center}
im Root-Verzeichnis des Quellcodes (das ist der Ordner, indem sich die \enquote{pom.xml} befindet) ausgeführt wird.
Eine JAR-Datei befindet sich dann im Ordner \enquote{target}.

    \chapter{Entwicklungsumgebung}\label{ch:entwicklungsumgebung}
\begin{table}[ht]
    \centering
    \label{tab:environment}
    \begin{tabular}{p{3.5cm}p{9cm}}
        \textbf{Betriebssystem} & \Betriebssystem\\
        \textbf{Hardware} & \Rechner~\CPU~24GB RAM\\
        \textbf{Compiler} & Java Development-Kit 19\\
    \end{tabular}
\end{table}

    \chapter{Verwendete Hilfsmittel}\label{ch:verwendete-hilfsmittel}

\begin{itemize}
    \item Visual Studio Code 1.76.2\\ Entwicklungsumgebung und Editor für Java und andere Programmiersprachen \\\url{https://code.visualstudio.com/}
    \item Maven\\Build-Tool für Java\\\url{https://maven.apache.org}
    \item TexLive 2023\\Softwarepaket für \LaTeX\\\url{https://miktex.org/}
    \item Structorizer\\Programm zur Erstellung von Nassi-Shneiderman-Diagrammen \\\url{https://structorizer.fisch.lu/}
    \item Visual Paradigm\\Programm zur Modellierung von Software-(Diagrammen) \\\url{https://www.visual-paradigm.com/}
    \item Git\\Versionsverwaltungssystem \\\url{https://git-scm.com/}
\end{itemize}

    \chapter{Erklärung}\label{ch:erklaerung}
Ich erkläre verbindlich, dass das vorliegende Prüfprodukt von mir selbstständig erstellt wurde.
Die als Arbeitshilfe genutzten Unterlagen sind der Arbeit vollständig aufgeführt.
Ich versichere, dass der vorgelegte Ausdruck mit dem Inhalt des von mir erstellten Datenträgers identisch ist.
Weder ganz noch in Teilen wurde die Arbeit bereits als Prüfungsleistung vorgelegt.
Mir ist bewusst, dass jedes Zuwiderhandeln als Täuschungsversuch zu gelten hat, der die Anerkennung des Prüfprodukts als Prüfungsleistung ausschließt.
\bigskip

\begingroup
\setlength{\parindent}{0pt} % keine Einrückung bei neuen Absätzen in diesem Bereich

\locationDocument, den \dateDocument
\bigskip
\bigskip

% gewünschte Breite der Unterschriftslinie
\newlength{\widthbox}
\settowidth{\widthbox}{\locationDocument, den \dateDocument}

\makebox[\widthbox]{\hrulefill}\\
\authorDocument
\endgroup

    \includepdf[scale=0.8, pages=1,pagecommand=\chapter{Aufgabenstellung}\label{ch:aufgabenstellung}, offset=0 -3cm]{images/Aufgabenstellung}
\includepdf[pages=2-]{images/Aufgabenstellung}

\end{document}
