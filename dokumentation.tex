\documentclass[
    12pt, % Schriftgröße
    oneside, % zweiseitiger Modus
    ngerman, % deutsches Dokument
    BCOR=0mm, % Bindungskorrektur
    DIV=10 % Division (Anzahl Spalten/Zeilen pro Seite, bestimmt implizit Margins)
]{scrreprt}

\newcommand{\ArbeitTitelseite}{Dokumentation der Praktischen Arbeit\ zur Prüfung zum\ Mathematisch-technischen Softwareentwickler}
\newcommand{\titleDocument}{\ArbeitTitelseite}

\newcommand{\Autor}{Leonhard Aaron Keßler}
\newcommand{\authorDocument}{\Autor}

\newcommand{\ArbeitKopfzeile}{Dokumentation der Praktischen Arbeit 2022}
\newcommand{\ArbeitThema}{Bestimmung der Minimalanzahl von Antennen}
\newcommand{\Pruefungsnummer}{TODO}
\newcommand{\Programmiersprache}{Java}
\newcommand{\Compiler}{Maven}
\newcommand{\Rechner}{Dell Precision 7520}
\newcommand{\Betriebssystem}{Windows 10}


\newcommand{\subjectDocument}{Großprogrammierung}
\newcommand{\locationDocument}{Köln}
\newcommand{\dateDocument}{\today}

\title{\titleDocument}
\author{\authorDocument}
\date{\dateDocument}

%Style importieren:
\usepackage{dokumentation}


\begin{document}
    % ============ Anfang =============
    % Titelseite
    \begin{titlepage}
	\thispagestyle{empty}

	\begin{flushright}
		\includegraphics[height=2cm]{images/axa_logo_open_blue_rgb}
	\end{flushright}

	\vspace{1.9cm}

	\begin{center}
		\rule{0.95\textwidth}{1pt}\\[.3cm]
		\begin{minipage}{0.9\textwidth}
			\renewcommand{\baselinestretch}{1.3}
			\begin{center}
				\LARGE \textbf{\ArbeitTitelseite}
			\end{center}
		\end{minipage}\\[.3cm]
		\rule{0.95\textwidth}{1pt}\\

		\vspace{2cm}

		\today

		\vspace{2cm}

		{\large \textbf{\authorDocument}}

		\vspace{2.0cm}

		\begin{tabular}{rl}
			Prüflingsnummer: & \Pruefungsnummer\\[.3cm]
			Programmiersprache: & \Programmiersprache\\[.3cm]
		\end{tabular}

		\vspace{1.9cm}

		\clearpage
		\thispagestyle{empty}
	\end{center}
\end{titlepage}


    \tableofcontents

    % =========== Zahlenteil ===========
    \chapter{Aufgabenanalyse}\label{ch:aufgabenanalyse}


\section{Interpretation der Aufgabe}\label{sec:interpretation-der-aufgabe}
Gefordert ist ein Programm, welches TODO\\ %TODO

Als Datenstruktur wird .\\%%TODO
Bevorzugt wird float, da 32-Bit mehr als ausreichend für Rechenoperationen mit Zahlen von bis zu 6.1.
Zudem kann durch Hardwarebeschleunigung, oder auch ohne, eine kürzere Laufzeit erreicht werden, im Gegensatz zu Double-Operationen.\\ %TODO

Die Eingabedatei wird zeilenweise eingelesen.
Beginnt eine Zeile mit einem Semicolon, wird diese als Kommentar beziehungsweise als Beschreibung interpretiert.
\begin{figure}[h]
    \centering
    \caption{Input-Restriktionen}
    \begin{itemize}[noitemsep]
        \item Restriktion 1.
    \end{itemize}
    \label{fig:input-restrictions}
\end{figure}

Die Lösung des Problems wird mittels TODO realisiert. %TODO
TODO Beschreibe Algorithmus %TODO


\section{Fehlerarten}\label{sec:fehlerarten}
Die Eingabedatei kann verschiedene Integritätsbedingungen verletzen.
Das Programm muss diese Fehlerarten identifizieren und den Nutzer darüber informieren.

\subsection{Technische Fehler}\label{subsec:technische-fehler}
%TODO

\subsection{Syntaktische Fehler}\label{subsec:syntaktische-fehler}
Die Eingabedatei muss der Struktur aus~\nameref{fig:input-restrictions} entsprechen.
So kann zum Beispiel ein syntaktischer Fehler provoziert werden, indem TODO%TODO

\subsection{Semantische Fehler}\label{subsec:semantische-fehler}


\section{Fehlerbehandlung}\label{sec:fehlerbehandlung}

\subsection{Technische Fehler}\label{subsec:technische-fehler-behandlung}

\subsection{Syntaktische Fehler}\label{subsec:syntaktische-fehler-behandlung}

\subsection{Semantische Fehler}\label{subsec:semantische-fehler-behandlung}
\subsection{Sonderfälle}\label{subsec:sonderfaelle}

    \chapter{Verfahrensbeschreibung}\label{ch:verfahrensbeschreibung}


\section{Gesamtsystem}\label{sec:gesamtsystem}
Das System arbeitet nach dem \textbf{EVA}-Prinzip.
Die einzelnen Komponenten laufen jeweils in Threads.
Die \textbf{EVA}-Segmente werden von einem Controller koordiniert, welcher gleichzeitig auch der Einstiegspunkt des Programms ist.
Zu Beginn des Programms nimmt der Controller per Argument einen Ordnerpfad entgegen.
Anschließend startet er die einzelnen Komponenten als Threads.

\subsection{Eingabe}\label{subsec:eingabe}
Sobald er gestartet wurde, liest der Eingabethread permanent Dateien aus dem übergebenen Ordner ein.
Er selbst führt eine Queue, welche alle bereits eingelesenen Dateien enthält.
Alle 0.05s wird von einem weiteren Thread, dessen einzige Aufgabe es ist, eine Methode aufzurufen ein, nach außen sichtbares, Model ersetzt.
Dadurch wird der gewünschte Effekt simuliert, dass der Detektor mit 20Hz immer eine andere Messreihe zur Verfügung stellt.
Der Controller schaut permanent, ob es ein sichtbares Objekt gibt und gibt dieses an die Verarbeitung weiter, falls sich das sichtbare Objekt geändert hat.

\subsection{Verarbeitung}\label{subsec:verarbeitung}
Die Verarbeitungskomponente führt ebenfalls eine Queue, welche von außen befüllt werden kann.
Gibt es ein neues Objekt zum Verarbeiten in der Queue, schaut sie, ob zu diesem Objekt bereits eine Ausgabe existiert.
Ist das nicht der Fall, wird der~\nameref{sec:mathematische-methoden}-Prozess gestartet.
Nach erfolgreicher Berechnung wird das Objekt an die Ausgabeinstanz weitergegeben.

\subsection{Ausgabe}\label{subsec:ausgabe}
Die Queue der Ausgabeinstanz wird von der Verarbeitung befüllt.
Gibt es ein neues Objekt und wurde dieses bisher noch nicht als Datei veröffentlicht, startet sie den Ausgabeprozess.
Hierbei wird das berechnete Modell in einer vorgegebenen Struktur in eine~.txt-Datei geschrieben.


\section{Strukturen}\label{subsec:strukturen}
Sowohl die Eingabe als auch die Ausgabe implementieren jeweils ein Interface, welches Runnable erweitert.
Durch das Interface sind implementierende Klassen gezwungen sowohl die Funktion einer Ein-/Ausgabe als auch die eines Runnables zur Verfügung zu stellen.
Runnables sind Objekte, die in einem Thread gestartet werden können.
Dies ermöglicht eine einfache Austauschbarkeit der Komponenten, welche im Controller initialisiert werden.

\subsection{Datenstruktur}\label{subsec:datenstruktur}
Eingelesene Dateien werden in einem Datensatz in einer Liste von Datenpunkten vom Typ Integer gespeichert.
Ein Datenpunkt ist ein generisches Objekt, welches drei Attribute hält.
Diese 3 Attribute sind vom selben Typ, welcher beim Erstellen eines Objekts festgelegt wird.
Im Datenpunkt werden die Position des Spiegels und die Intensität gespeichert.
Nach der Verarbeitung eines Datensatzes enthält dieser eine weitere Liste an Datenpunkten vom Typ Double.
Diese Datenpunkte haben als drittes Attribut zusätzlich den Wert der oberen Einhüllenden befüllt.
Im Datensatz werden außerdem Dateiname, Pulsbreite und das Maximum der Intensität gespeichert.\\
Siehe auch:
\protect\newpage
\section{Mathematische Methoden}\label{sec:mathematische-methoden}

    \chapter{Programmbeschreibung}\label{ch:programmbeschreibung}

\section{Struktogramme}\label{sec:structogram}
\begin{center}
    \makebox[\textwidth]{\includesvg[width=\paperwidth]{GroPro-Leonhard}}
\end{center}

\section{Entwicklungsdokumentation}\label{sec:entwicklerdokumentation}
Die Dokumentation des Programms wurde in Javadoc vorgenommen und kann im Ordner javadoc eingesehen werden.
Hierzu kann die index.html aufgerufen werden
%    \addtocontents{toc}{\protect\newpage}
    \chapter{Testdokumentation}\label{ch:testdokumentation}


\section{Normalfälle}\label{sec:normalfaelle}
\section{Fehlerfälle}\label{sec:fehlerfaelle}
\section{Grenzfälle}\label{sec:grenzfaelle}
\section{Sonderfälle}\label{sec:sonderfaelle}

    \chapter{Zusammenfassung und Ausblick}\label{ch:zusammenfassung-und-ausblick}


\section{Zusammenfassung}\label{sec:zusammenfassung}

\section{Ausblick}\label{sec:ausblick}


    % ============= Buchstabenteil ==============
    \renewcommand{\thechapter}{\Alph{chapter}}%
    \setcounter{chapter}{0}
    \chapter{Abweichung und Ergänzung}\label{ch:abweichung-und-ergaenzung}
    \chapter{Benutzeranleitung}\label{ch:benutzeranleitung}


\section{Vorbereiten des Systems}\label{sec:vorbereiten-des-systems}

\subsection{Systemvoraussetzungen}\label{subsec:systemvoraussetzungen}
Um sicherzustellen, dass das Programm lauffähig ist, sollte~\Betriebssystem~als Betriebssystem genutzt werden.
Es ist außerdem eine JRE oder JDK in der Version 17 oder höher vonnöten, um das Programm auszuführen.\\
Falls eine erneute Kompilierung des Programms gewünscht ist, empfiehlt es sich die JDK anstelle der JRE zu installieren.
Um diese JRE/JDK anschließend zu nutzen, muss das bin-Verzeichnis dieser in die PATH-Umgebungsvariable hinzugefügt werden.
\subsection{Installation}\label{subsec:installation}
Es ist keine Installation nötig, es reicht das~.zip-Verzeichnis zu entpacken.

\section{Programmaufruf}\label{sec:programmaufruf}
Nach dem Entpacken kann das Programm über den Befehl
\begin{center}
    \colorbox{gray!20}{
        \begin{minipage}{0.9\textwidth}
            java -jar GroPro-1.0.jar "Testbeispiele/Test1IHK.txt"
        \end{minipage}
    }
\end{center}
in der Eingabeaufforderung (CMD) oder beliebiger Bash ausgeführt werden.
\section{Testen der Beispiele}\label{sec:testen-der-beispiele}
Das Ausführen der automatischen Tests erfolgt über die Datei \enquote{RunTestBeispiele.cmd}.
In der Konsole wird nun das Programm für jede Datei im Ordner \enquote{Testbeispiele} ausgeführt.
Alle Dateiausgaben befinden sich im Anschluss im Ordner \enquote{Testbeispiele/out}.

\section{Kompilieren}\label{sec:kompilieren}
Zum Erzeugen der~.jar-Datei sollte Maven genutzt werden.
Der Einfachheit halber muss das bin-Verzeichnis der Maveninstallation ebenfalls in der PATH-Umgebungsvariable aufgenommen werden.
Anschließend lässt sich das Programm kompilieren, indem der Befehl
\begin{center}
    \colorbox{gray!20}{
        \begin{minipage}{0.9\textwidth}
            mvn package
        \end{minipage}
    }
\end{center}
im Root-Verzeichnis des Quellcodes (das ist der Ordner, indem sich die \enquote{pom.xml} befindet) ausgeführt wird.
Eine JAR-Datei befindet sich dann im Ordner \enquote{target}.
    \chapter{Entwicklungsumgebung}\label{ch:entwicklungsumgebung}
\begin{table}[ht]
    \centering
    \label{tab:environment}
    \begin{tabular}{p{3.5cm}p{9cm}}
        \textbf{Betriebssystem} & \Betriebssystem\\
        \textbf{Hardware} & \Rechner~\CPU~24GB RAM\\
        \textbf{Compiler} & Java Development-Kit 19\\
    \end{tabular}
\end{table}

    \chapter{Verwendete Hilfsmittel}\label{ch:verwendete-hilfsmittel}

\begin{itemize}
    \item Visual Studio Code 1.76.2\\ Entwicklungsumgebung und Editor für Java und andere Programmiersprachen \\\url{https://code.visualstudio.com/}
    \item Maven\\Build-Tool für Java\\\url{https://maven.apache.org}
    \item TexLive 2023\\Softwarepaket für \LaTeX\\\url{https://miktex.org/}
    \item Structorizer\\Programm zur Erstellung von Nassi-Shneiderman-Diagrammen \\\url{https://structorizer.fisch.lu/}
    \item Visual Paradigm\\Programm zur Modellierung von Software-(Diagrammen) \\\url{https://www.visual-paradigm.com/}
    \item Git\\Versionsverwaltungssystem \\\url{https://git-scm.com/}
\end{itemize}

    \chapter{Erklärung}\label{ch:erklaerung}
Ich erkläre verbindlich, dass das vorliegende Prüfprodukt von mir selbstständig erstellt wurde.
Die als Arbeitshilfe genutzten Unterlagen sind der Arbeit vollständig aufgeführt.
Ich versichere, dass der vorgelegte Ausdruck mit dem Inhalt des von mir erstellten Datenträgers identisch ist.
Weder ganz noch in Teilen wurde die Arbeit bereits als Prüfungsleistung vorgelegt.
Mir ist bewusst, dass jedes Zuwiderhandeln als Täuschungsversuch zu gelten hat, der die Anerkennung des Prüfprodukts als Prüfungsleistung ausschließt.
\bigskip

\begingroup
\setlength{\parindent}{0pt} % keine Einrückung bei neuen Absätzen in diesem Bereich

\locationDocument, den \dateDocument
\bigskip
\bigskip

% gewünschte Breite der Unterschriftslinie
\newlength{\widthbox}
\settowidth{\widthbox}{\locationDocument, den \dateDocument}

\makebox[\widthbox]{\hrulefill}\\
\authorDocument
\endgroup

    \includepdf[scale=0.78, pages=2,pagecommand=\chapter{Aufgabenstellung}\label{ch:aufgabenstellung}, offset=0 -3cm]{images/Aufgabenstellung}
\includepdf[pages=3-]{images/Aufgabenstellung}
    \chapter{Quellcode}\label{ch:quellcode}
    \chapter{In- und Output der Testdokumentation}\label{ch:in-out}
\end{document}